% Options for packages loaded elsewhere
\PassOptionsToPackage{unicode}{hyperref}
\PassOptionsToPackage{hyphens}{url}
%
\documentclass[
]{article}
\usepackage{amsmath,amssymb}
\usepackage{lmodern}
\usepackage{iftex}
\ifPDFTeX
  \usepackage[T1]{fontenc}
  \usepackage[utf8]{inputenc}
  \usepackage{textcomp} % provide euro and other symbols
\else % if luatex or xetex
  \usepackage{unicode-math}
  \defaultfontfeatures{Scale=MatchLowercase}
  \defaultfontfeatures[\rmfamily]{Ligatures=TeX,Scale=1}
\fi
% Use upquote if available, for straight quotes in verbatim environments
\IfFileExists{upquote.sty}{\usepackage{upquote}}{}
\IfFileExists{microtype.sty}{% use microtype if available
  \usepackage[]{microtype}
  \UseMicrotypeSet[protrusion]{basicmath} % disable protrusion for tt fonts
}{}
\makeatletter
\@ifundefined{KOMAClassName}{% if non-KOMA class
  \IfFileExists{parskip.sty}{%
    \usepackage{parskip}
  }{% else
    \setlength{\parindent}{0pt}
    \setlength{\parskip}{6pt plus 2pt minus 1pt}}
}{% if KOMA class
  \KOMAoptions{parskip=half}}
\makeatother
\usepackage{xcolor}
\usepackage[margin=1in]{geometry}
\usepackage{graphicx}
\makeatletter
\def\maxwidth{\ifdim\Gin@nat@width>\linewidth\linewidth\else\Gin@nat@width\fi}
\def\maxheight{\ifdim\Gin@nat@height>\textheight\textheight\else\Gin@nat@height\fi}
\makeatother
% Scale images if necessary, so that they will not overflow the page
% margins by default, and it is still possible to overwrite the defaults
% using explicit options in \includegraphics[width, height, ...]{}
\setkeys{Gin}{width=\maxwidth,height=\maxheight,keepaspectratio}
% Set default figure placement to htbp
\makeatletter
\def\fps@figure{htbp}
\makeatother
\setlength{\emergencystretch}{3em} % prevent overfull lines
\providecommand{\tightlist}{%
  \setlength{\itemsep}{0pt}\setlength{\parskip}{0pt}}
\setcounter{secnumdepth}{-\maxdimen} % remove section numbering
\ifLuaTeX
  \usepackage{selnolig}  % disable illegal ligatures
\fi
\IfFileExists{bookmark.sty}{\usepackage{bookmark}}{\usepackage{hyperref}}
\IfFileExists{xurl.sty}{\usepackage{xurl}}{} % add URL line breaks if available
\urlstyle{same} % disable monospaced font for URLs
\hypersetup{
  pdftitle={Science Communication},
  hidelinks,
  pdfcreator={LaTeX via pandoc}}

\title{Science Communication}
\author{}
\date{\vspace{-2.5em}}

\begin{document}
\maketitle

\emph{This page will be updated with ongoing and upcoming Science
Communication work.}

\hypertarget{the-dorsal-column}{%
\subsection{The Dorsal Column}\label{the-dorsal-column}}

The Dorsal Column is a quarterly publication sharing science
communication pieces related to brain research. Each issue features
articles written and reviewed by a team of graduate students and
postdoctoral fellows in affiliation with the Society of Neuroscience
Graduate Students at Western University. \textbf{Recent pieces for the
\href{https://songsuwo.ca/dc-about-us}{The Dorsal Column}:}

\textbf{The Damaging Effects of Orphanages on the Brain} Written by
Michaela Kent
\href{https://songsuwo.ca/thedorsalcolumn/vol3-iss2-michaela-kent}{Read
the article here}

\textbf{The amygdala, anxiety and Autism Spectrum Disorder} Written by
Michaela Kent
\href{https://songsuwo.ca/thedorsalcolumn/vol3-iss4-michaela-kent}{Read
the article here}

\textbf{The Promise of Bedside Brain Imaging in Babies} Written by
Michaela Kent
\href{https://songsuwo.ca/thedorsalcolumn/vol4-iss2-michaela-kent}{Read
the article here}

\hypertarget{inspiring-minds}{%
\subsection{Inspiring Minds}\label{inspiring-minds}}

Michaela was selected to be featured in Western University's Inspiring
Minds showcase. Inspiring Minds seeks to broaden awareness and impact of
graduate student research, while enhancing transferable skills. Read
about how Michaela's work is using optical neuroimaging to understand
social interactions \href{https://fal.cn/3x5T4}{here}.

\hypertarget{podcast}{%
\subsection{Podcast}\label{podcast}}

Michaela featured as a guest speaker on the podcast Brain Matter
Chatter, a podcast that aims to raise awareness about issues surrounding
mental health in academia. For more information or to hear other
episodes, please visit their website
\href{https://songsuwo.ca/brainmatterchatter}{here}.

\emph{``In this episode, Michaela Kent, a PhD student in neuroscience,
at Western University, joins Ruby and Julia for a discussion on Zoom
fatigue. Michaela draws on her expertise in, and research on, virtual
socialization to explain why Zoom fatigue leaves many of us feeling
unmotivated and unfilled at the end of the day. As an international
student, Michaela shares her experiences of being a''Zoom student'' in
the midst of the pandemic.''}

\hypertarget{art-science-collaborations}{%
\subsection{Art-Science
Collaborations}\label{art-science-collaborations}}

\textbf{Translating Mind into Matter}

This was part of the Art-Science Collaboration hosted at
\href{https://songsuwo.ca/nrd2023}{Neuroscience Research Day, 2023}.
This is a unique opportunity for Neuroscience researchers to pair up
with an artist and communicate their research through art. For more
information and to see other artwork submissions please visit the
website \href{https://songsuwo.ca/brain-art}{here}. Jessica Joyce is an
MFA student in the Visual Arts Department at Western University. She is
a representational painter whose current focus rests on autobiographical
exploration of climate change and its link to mental health. Her thesis
work explores her own position as a white settler in Canada through the
language of self-portraiture. For this project she will wear a
neuroimaging cap constructed by Michaela while painting a self-portrait,
using a mirror to be able to work from direct observation. For Jessica,
this collaboration with Michaela is another medium through which to
study the process and product of painting in the context of mental
health. She knows how she feels when painting and is curious to see how
the story of Michaela's data might enhance her understanding of her own
lived experiences. To complement the painting, we used this
collaboration as an opportunity to explore brain activity underlying the
introspective process of self-portraiture. Using brain imaging
techniques (functional near-infrared spectroscopy), Michaela will create
a map of the active brain regions during the session. Near-infrared
light is shone into the brain to measure blood flow, giving an
indication of brain activation. Michaela, a PhD student in the
Neuroscience program, uses this technique to study social interactions
and the developing brain. For this, she believes that face-to-face
communication is key and her research strives to allow neuroscience
research to occur in more naturalistic settings. The painting and brain
maps from the session were displayed side-by-side in order to
contextualize each other. \textbf{Artist: Jessica Joyce}
\emph{(Department of Visual Arts, Faculty of Arts \& Humanities, Western
University)}

\textbf{Scientist: Michaela Kent} \emph{(Western Institute for
Neuroscience, Schulich School of Medicine \& Dentistry, Western
University)}

View this post on Instagram

A post shared by Jessica Joyce (@jessicairenejoyce)

\textbf{Human Connection}

This image was created as part of the Art-Science Collaboration hosted
at \href{https://songsuwo.ca/nrd2022}{Neuroscience Research Day, 2022}.
Functional near-infrared spectroscopy (fNIRS) is a type of optical
neuroimaging that is becoming an increasingly popular method of studying
human brain function. Near-infrared light is shone into the brain to
measure blood flow, giving an indication of brain activation during
different tasks or events. Its advantages over other techniques include
that it allows for hyperscanning, or the simultaneous imaging of two
people's brains as they interact. This allows researchers to look at
brain synchrony, often characterized by similar patterns of activity.
This is represented in the artwork where lines appear to ``connect'' the
brains, whilst also being representative of the physical fNIRS set up.
Likewise, the bright colours are representative of the activation seen
in different parts of the brain as areas ``light up'' when we
communicate. Michaela, a PhD student in the Neuroscience program, uses
fNIRS to study social interactions and the developing brain. For this,
she believes that face-to-face communication is key and her research
strives to allow neuroscience research to occur in more naturalistic
settings. Bringing together art and neuroscience, Audra has created a
piece that highlights some key aspects of brain research. The piece was
inspired by the process of communication and how we interact with one
another. Using images and references from Michaela's research, Audra was
able to source inspiration that ties human connection together.
\textbf{Artist: Audra Bartel} \emph{(Department of Visual Arts, Faculty
of Arts \& Humanities, Western University)}

\textbf{Scientist: Michaela Kent} \emph{(Western Institute for
Neuroscience, Schulich School of Medicine \& Dentistry, Western
University)}

\end{document}
